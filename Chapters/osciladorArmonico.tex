\chapter{Oscilador Armónico Cuántico}

\section{Problemas}

\fbox{ \begin{minipage}{\textwidth}
\textbf{Ejercicio 1:} 
\begin{itemize}
	\item Cuanto avanza una onda armónica en un periódo? 
	\item Cuanto tarda a desplacarse a una distancia igual a la longitud de la
onda?
	\item La longitud de una onda de la nota musical LA a la aire es de 0.773 m.
Cuales son la su frequencia y su longitud de onda en el agua? La velocidad del
sonido en el aire es 340 m/s y en el agua 1.44 km/s.
\end{itemize}
\end{minipage}} \\\\
\textbf{Solucion:} \\
\begin{itemize}
	\item Una onda avanza una distancia de una longitud de una onda
\textbf{$\lambda$} en una periodo \textbf{$T$}. 
	$$
		v_{onda} = \frac{s}{T} = \lambda \cdot f = \lambda \cdot \frac{1}{T}  \quad
\Rightarrow \qquad s = \lambda
	$$
	\item Igual que antes: \textbf{T}.
	\item Primero queremos encontrar a la frequencia del aire $f_{aire}$, que
esta dado por
	$$
		\lambda_{aire} = \frac{v_{aire}}{f_{aire}} \quad \Rightarrow \qquad f_{aire} =
\frac{v_{aire}}{\lambda_{aire}} = \frac{340 m/s}{0,773m} \approx 440 Hz
	$$
	Por los dos medias (aire y agua) la frequencia es el mismo
	$$
		f_{aire} = f_{agua}
	$$
	por eso podemos calcular la longitud de la onda de la onda en el agua
	$$
		\lambda_{agua} = \frac{v_{agua}}{f_{agua}} = \frac{1440 m/s}{440 s}
\approx 3,27 m
	$$
\end{itemize}

\begin{ejercicio}
	Dado la onda armónica 
	$$
		y(x, t) = 2 \sin(\pi x - 20 \pi t)
	$$
	con $y$ en cm, $x$ en m y $t$ en segundos, evaluar
	\begin{itemize}
		\item el sentido de la propagación
		\item la amplituda de la onda $A$, su longitud $\lambda$, su frequencia $f$,
	\end{itemize}
su periodo $T$ y su velocidad de propagación $v$
\end{ejercicio}
\begin{itemize}
	\item 
	La forma general de una onda esta dado por
	\begin{align*}
		y(x, t) &= A \sin(kx - \omega t) \quad \text{si la onda mueve a la derecha} \\
		y(x, t) &= A \sin(kx + \omega t) \quad \text{si la onda mueve a la
	izquierda}
	\end{align*}
	Por eso podemos decir que la onda tiene un sentido de la propagación
\textbf{a la derecha}.
	\item 
	Con las relación general de una onda moviendo a la deracha podemos
facilmente leer las parametras de la amplituda $A$, el numeró de la honda $k$ y
la velocidad angular $\omega$
	\begin{align*}
		y(x, t) &= A \sin(kx - \omega t) \\
		y(x, t) &= 2 \sin(\pi x - 20 \pi t)
	\end{align*}
	$$
		\Rightarrow \quad A = 2cm, \quad k = \pi, \quad \omega = 20 \pi
	$$
	Con esos parametos podemos calcular el resto:
	\begin{align*}
		\omega = \frac{2 \pi}{T} = 2\pi f = 2\pi \frac{v}{\lambda} \\
		k = \frac{2\pi}{\lambda} \quad\Rightarrow \quad \lambda = \frac{2 \pi}{\pi 1/m} =
2m \\
	\omega = 2\pi \frac{v}{\lambda} \quad\Rightarrow \quad v =
\frac{\omega}{2\pi} \lambda = \frac{20 \pi 1/s}{2\pi} 2m = 20 m/s \\
	f = \frac{v}{\lambda} = \frac{20 m/s}{2m} = 10 Hz \quad \Rightarrow \quad T
= \frac{1}{f} = 0,1 s 
	\end{align*}	
\end{itemize}

\begin{ejercicio}
	\textbf{Ejercicio 3} Un corcho esta oscillando en el agua con una velocidad
vertical maximal de $3cm /s$ y una accerlación máxima de $2cm /s^2$. 
	\begin{itemize}
		\item Calcula la amplitude y la frecuencia del movimiento.
		\item Si la longitud de la onda transversal generada es de 3m, cual es
la velocidad de la propagación de la onda?	
	\end{itemize}
\end{ejercicio}
\begin{itemize}
	\item
	Para ese ejecicio tenemos que saber dos relaciones sobre la velocidad y la
acceleración maxima
	\begin{align*}
		\left.
		\begin{aligned}
			v_{max} = w A \quad \Rightarrow \quad w = \frac{v_{max}}{A} \\
			a_{max} = w^2 A  \quad \Rightarrow \quad \frac{v_{max}^2}{a_{max}} =
a_{max}
		\end{aligned}
		\right\rbrace
		\quad \Rightarrow \quad A = \frac{v_{max}^2}{a_{max}} = \frac{9
cm^2/s^2}{2 cm/s^2} = 4,5 cm
	\end{align*}
	Tambien tenemos una relacion por la velocidad angular y por eso tambien
tenemos la frecuencia
	$$
		\omega = \frac{v_{max}}{A} = \frac{2}{2} 1/s \quad \Rightarrow \quad f =
\frac{\omega}{2\pi} = \frac{1}{3 \pi} 1/s
	$$
	
	\item
	Si tenemos la longitud de la onda su velocidad esta dado por
	$$
		f = \frac{v}{\lambda} \quad \Rightarrow \quad v = f \lambda =
\frac{1}{\pi} cm / s = 0,318 cm /s
	$$
\end{itemize}
