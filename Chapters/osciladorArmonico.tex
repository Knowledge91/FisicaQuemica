\chapter{Oscilador Armónico Cuántico}

\section{Problemas}

\fbox{ \begin{minipage}{\textwidth}
\textbf{Ejercicio 1:} 
\begin{itemize}
	\item Cuanto avanza una onda armónica en un periódo? 
	\item Cuanto tarda a desplacarse a una distancia igual a la longitud de la
onda?
	\item La longitud de una onda de la nota musical LA a la aire es de 0.773 m.
Cuales son la su frequencia y su longitud de onda en el agua? La velocidad del
sonido en el aire es 340 m/s y en el agua 1.44 km/s.
\end{itemize}
\end{minipage}} \\\\
\textbf{Solucion:} \\
\begin{itemize}
	\item Una onda avanza una distancia de una longitud de una onda
\textbf{$\lambda$} en una periodo \textbf{$T$}. 
	$$
		v_{onda} = \frac{s}{T} = \lambda \cdot f = \lambda \cdot \frac{1}{T}  \quad
\Rightarrow \qquad s = \lambda
	$$
	\item Igual que antes: \textbf{T}.
	\item Primero queremos encontrar a la frequencia del aire $f_{aire}$, que
esta dado por
	$$
		\lambda_{aire} = \frac{v_{aire}}{f_{aire}} \quad \Rightarrow \qquad f_{aire} =
\frac{v_{aire}}{\lambda_{aire}} = \frac{340 m/s}{0,773m} \approx 440 Hz
	$$
	Por los dos medias (aire y agua) la frequencia es el mismo
	$$
		f_{aire} = f_{agua}
	$$
	por eso podemos calcular la longitud de la onda de la onda en el agua
	$$
		\lambda_{agua} = \frac{v_{agua}}{f_{agua}} = \frac{1440 m/s}{440 s}
\approx 3,27 m
	$$
\end{itemize}






