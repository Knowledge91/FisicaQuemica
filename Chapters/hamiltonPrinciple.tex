\chapter{Variational Principle and Lagrange's Equations}
In the last chapter we got to know a new principle of classical mechanics,
which equivalently to \textit{Newton mechanic} led us to the equations of
motion. In general the laws of classical mechanics can be described within two
\textbf{Variational principles}.
\begin{itemize}
  \item Differential principle (D'Alembert)
  \item Integral principle (Hamilton)
\end{itemize}
Within the \textit{d'Alembert principle} we compare the instanteneous state of
a system with its infinitesimal virtual displacements and get the equations of
motion as a result. Within the \textit{Hamilton principle} we consider the 
entire \textit{motion of the system}\footnote{Term will be described in the
section of Hamilton's principle.} between times $t_1$ and $t_2$ and small virtual 
variations of this motion, leading aswell to the equations of motion. 

The development of the second approach will be topic of this chapter.

\section{Hamilton principle}
As we have learned we can describe the state of our system, containing N
particles with the \textit{generalized coordinates} $q_1, \cdots, q_N$. The
space describing our state in \textit{generalized coordinates} is calles
\textit{configuration space}. As time goes on the time dependent
\textit{generalized coordinates} $q_1(t), \cdots, q_N(t)$ will change, and so
will our point in \textit{configuration space}, describing a curve called
\textit{the path of motion of the system}.
\begin{equation}
  \vec q(t) \,=\, (q_1(t), \cdots, q_N(t))
\end{equation}

For the following derivation we will only consider \textit{conservative holonomic
systems}. Plugin in the time dependent \textit{generalized coordinates} into
our \textit{Lagrangian} we get a explicitly time dependent function $\tilde L$
\begin{equation}
  L(\vec q(t), \dot{\vec q}(t), t) \,=\, \tilde L(t) 
\end{equation}
Thus we can define a functional\footnote{A functional is simply a function
depending on a function. In our case the function S depending on the function
$\vec q(t)$.}
\begin{equation}
  S[\vec q(t)] \,=\, \int_{t_1}^{t_2} \tilde L(t),
\end{equation}
which is referred to as \textit{the action} and dependend on $\vec q(t)$ and the t.
For fixed $t_1$ and $t_2$ the functional $S(\vec q(t))$ maps a value to every
$q(t)$. Consequently there are many \textit{paths of motion of the system}
connecting the intial state $\vec q(t_1)$ with the final state $\vec q(t_2)$.
Our aim is now to variate the paths and find the optimized
\textit{path of motion for the system}. Hence we are dealing with an optimization 
problem. Introducing virtual displacements $\delta \vec q(t)$, which are zero at the 
end points $\delta \vec q(t_1) \,=\, \delta \vec q(t_2) \,=\, 0$\footnote{Every $\vec
q(t)$ represents a state in the \textit{configuration space}. As the initial
and the final state are fixed, the variation at the end points $\vec q(t_1),
\vec q(t_2)$ have to be fixed to.}. We can now define \textit{Hamilton's
principle} 
\begin{equation}
  \delta S = \delta \int_{t_1}^{t_2} L(\vec q(t), \dot{\vec q}(t), t) dt \,=\, 0
\end{equation}
which states, that for all possible \textbf{variations} ($\delta \vec q(t)_i$) the system will choose the
one leading to a \textit{path of motion} for which \textit{the action} is
stationary!

Our next task will be to find a way using \textit{Hamilton's principle}.
Therefore we have to have a deeper insight in the \textit{Variational
Calculus}.

\section{Lagrange's Equation from the Action Integral}
Our aim is to find an \textit{action} for which the \textit{path of motion} is
\textit{stationary}. To simplify the introduction we will consider a one
dimensional Lagrangian $L(q, \dot q, t)$, implicitly depending on t. Let $q(t)$
represent a transition in the states $q_1 \,=\, q(t_1)$ to $q_2 \,=\, q(t_2)$. We
consider now a variation in the path, introducing a small pertubation
$\epsilon(t)$, which is zero at the endpoints $t_1$ and $t_2$
\begin{equation}
  q(t) \longrightarrow q(t) + \epsilon 
\end{equation}
Consequently the first order changes in our integral action are given by
\begin{equation}
  \begin{aligned}
  \delta S[q] \,=\,& S[q + \epsilon] - S[q] \\
  \,=\,& \int_{t_1}^{t_2} L(q + \epsilon, \dot q + \dot \epsilon, t) dt -
\int_{t_1}^{t_2} L(q, \dot q, t) dt.
  \end{aligned}
\end{equation}
Using the \textit{multivariable Taylor expansion} on L yields
\begin{equation}
  L(q + \epsilon, \dot q + \dot \epsilon, t) \,=\, L(q, \dot q, t) + \frac{L(q,
\dot q, t)}{q} \epsilon + \frac{L(q, \dot q, t)}{\dot q} \dot \epsilon +
\mathcal O(\epsilon^2).
\end{equation}
Hence our variated action integral can be written as 
\begin{equation}
  \delta S[q] \,=\, \int_{t_1}^{t_2} \left[\frac{\partial L}{\partial q} \epsilon +
\frac{\partial L}{\partial \dot q} \dot \epsilon  \right] dt
\end{equation}
while restricting our self to the first order of the previously used
\textit{Taylor expansion}. Regarding the second term in the integral we can
perform a \textit{integration by parts}
\begin{equation}
  \int_{t_1}^{t_2} \,=\, \frac{\partial L}{\partial \dot q} \dot \epsilon \,=\,
\int_{t_1}^{t_2} \frac{\partial L}{\partial \dot q}\frac{\partial
\epsilon}{\partial t} \,=\, \left. \frac{\partial L}{\partial q} \epsilon
\right|_{t_1}^{t_2} - \int_{t_1}^{t_2} \frac{d}{d t} \frac{\partial L}{\partial 
\dot q} \epsilon dt .
\end{equation}
As $\epsilon(t_1) \,=\, \epsilon(t_2) \,=\, 0$ the first tirm of the right-hand
side vanishes and we can rewrite our \textit{action integral} into
\begin{equation}
  \delta S[q] \,=\, \int_{t_1}^{t_2} \left[\frac{\partial L}{\partial q} -
\frac{d}{dt} \frac{\partial L}{\partial \dot q}\right]\epsilon dt  
\end{equation}
Using the \textit{fundamental Lemma} we can directly read out
\textit{Lagrange's equations}
\begin{equation}
  \frac{\partial L}{\partial q} - \frac{d}{dt} \frac{\partial L}{\partial \dot
q }\,=\, 0
\end{equation}


\section{Example}
A particle with the mass m is located in the gravitational field of the earth.
It has an one dimensional movent in z direction. Calculate the action integral
\begin{equation}
  S \,=\, \int_{t_1}^{t_2} L(z, \dot z, t) dt
\end{equation}
for a path 
\begin{equation}
  z(t) \,=\, - \frac{1}{2} gt^2 + f(t).
\end{equation}
The function $f(t)$ shall be an arbitraly, continous function with $f(t_1)
\,=\, f(t_2) \,=\, 0$. Show that the action gets minimal for $f(t) \,\equiv\, 0$.

As our action integral contains the Lagrangian we will start by formulating the
kinetic and potential energy. The movement is only in z-direction with a
gravitational force in the same direction. Hence 
\begin{equation*}
  T \,=\, \frac{1}{2} m \dot z^2, \quad V \,=\, mgz \quad \Rightarrow \quad L
\,=\, \frac{1}{2} m \dot z^2 - mgz.
\end{equation*}
In addition the path z(t) is given in the exercise
\begin{align*}
  z(t) \,=\,& -\frac{1}{2} gt^2 + f \\
  \dot z(t) \,=\,& -gt + \dot f.
\end{align*}
Thus we can write our action integral as
\begin{equation*}
  \begin{aligned}
    S \,=\,& \int_{t_1}^{t_2} L(z, \dot z) dt
    \,=\,& \int_{t_1}^{t_2} \frac{1}{2} m \dot z^2 - mgz \\
    \,=\,& \int_{t_1}^{t_2} dt [ \frac{1}{2} m (-gt + \dot f)^2] - \int_{t_1}^{t_2}
dt [mg (-\frac{1}{2}
gt^2 + f)] \\
    \,=\,& \int_{t_1}^{t_2} dt (- \frac{m}{2} g^2 t^2 + \frac{m}{2} \dot f^2 -
m g t \dot f + \frac{m}{2} g^2 t^2 - mgf) \\
    \,=\,& mg \int_{t_1}^{t_2} dt t^2 + \frac{m}{2} \int_{t_1}^{t_2} dt \dot
f^2 - mg \int_{t_1}^{t_2} dt (t \dot f + f) 
  \end{aligned}
\end{equation*}
Perfoming a partial integration 
\begin{equation*}
  \int_{t_1}^{t_2} dt t \dot f \,=\, t f |_{t_1}^{t_2} - \int_{t_1}^{t_2} dt f
\end{equation*}
we notice that the $tf |_{t_1}^{t_2}$ vanishes (because $f(t_1) \,=\, f(t_2)
\,=\, 0$) vanishes with the last term of our action integral. Hence we are left
with
\begin{equation*}
  \begin{aligned}
    S \,=\,& mg^2 \int_{t_1}^{t_2} dt t^2 + \frac{m}{2} \int_{t_1}^{t_2} dt
\dot f^2 \\
    \,=\,& mg \frac{1}{3} (t_2^3 t_1^3) + \frac{m}{2} \int_{t_1}^{t_2} dt \dot
f^2,
  \end{aligned}
\end{equation*}
which will be our final expression for the action integral.

Now that we have an expression for our action integral we want to show that it
gets minimal for $f \,\equiv\, 0$. Regarding our action integral we notice that
the first term is independent of f and that the second term gets minimal for
$\dot f(t) \,=\, 0$. As f has to fixed points $f(t_1) \,=\, f(t_2) \,=\, 0$ and
its slope is zero for a minimal action we follow that $f(t) \,=\, 0$ for a minimal
action. 
